\documentclass[a4paper,oneside,12pt]{extreport}

\include{preamble}


\begin{document}

\include{title}

\section*{Практическая часть}
В файле '/usr/include/x86\_64-linux-gnu/bits/types/FILE.h' было создано дополнительное 
имя для структуры FILE, которая используется в нашей программе.
\begin{lstlisting}[language=C]
typedef struct _IO_FILE FILE;
\end{lstlisting}

В файле 'cat /usr/include/x86\_64-linux-gnu/bits/libio.h' находится описание структуры \_IO\_FILE
\begin{lstlisting}[language=C]
struct _IO_FILE
{
int _flags; /* High-order word is _IO_MAGIC; rest is flags. */
#define _IO_file_flags _flags

/* The following pointers correspond to the C++ streambuf protocol. */
/* Note:  Tk uses the _IO_read_ptr and _IO_read_end fields directly. */
char *_IO_read_ptr;   /* Current read pointer */
char *_IO_read_end;   /* End of get area. */
char *_IO_read_base;  /* Start of putback+get area. */
char *_IO_write_base; /* Start of put area. */
char *_IO_write_ptr;  /* Current put pointer. */
char *_IO_write_end;  /* End of put area. */
char *_IO_buf_base;   /* Start of reserve area. */
char *_IO_buf_end;    /* End of reserve area. */
/* The following fields are used to support backing up and undo. */
char *_IO_save_base;   /* Pointer to start of non-current get area. */
char *_IO_backup_base; /* Pointer to first valid character of backup area */
char *_IO_save_end;    /* Pointer to end of non-current get area. */

struct _IO_marker *_markers;

struct _IO_FILE *_chain;

int _fileno;
#if 0
int _blksize;
#else
int _flags2;
#endif
_IO_off_t _old_offset; /* This used to be _offset but it's too small.  */

#define __HAVE_COLUMN /* temporary */
/* 1+column number of pbase(); 0 is unknown. */
unsigned short _cur_column;
signed char _vtable_offset;
char _shortbuf[1];

/*  char* _save_gptr;  char* _save_egptr; */

_IO_lock_t *_lock;
#ifdef _IO_USE_OLD_IO_FILE
};
\end{lstlisting}
\newpage
\begin{task}
    Первая программа один поток.
    \begin{lstlisting}[language=C]
#include <stdio.h>
#include <fcntl.h>

#define FILE_NAME "alphabet.txt"
#define BUFF_SIZE 20

#define GREEN "\33[32m"
#define BLUE "\33[34m"

int main()
{
	int fd = open(FILE_NAME, O_RDONLY); 
	
	FILE *fs1 = fdopen(fd, "r");
	char buff1[BUFF_SIZE];
	
	FILE *fs2 = fdopen(fd, "r");
	char buff2[BUFF_SIZE];
	setvbuf(fs2, buff2, _IOFBF, BUFF_SIZE);
	
	printf("\nfs1 _fileno: %d\n", fs1->_fileno);
	printf("\nfs2 _fileno: %d\n", fs2->_fileno);
	printf("\nfs1 buff1[0] == fs1->_IO_buf_base: %d\n", buff1 == fs1->_IO_buf_base);
	printf("\nfs2 buff2[0] == fs2->_IO_buf_base: %d\n", buff2 == fs2->_IO_buf_base);
	
	int flag1 = 1, flag2 = 2;

	while (flag1 == 1 || flag2 == 1)
	{
		char c;
		flag1 = fscanf(fs1, "%c", &c);
		if (flag1 == 1)
		{
			fprintf(stdout, GREEN "%c", c);
		}
		flag2 = fscanf(fs2, "%c", &c);
		if (flag2 == 1)
		{
			fprintf(stdout, BLUE "%c", c);
		}
	}
	
	printf("\n");
	return 0;
}
    \end{lstlisting}

    %\newpage

    \begin{center}
        \textbf{Объяснение}        
    \end{center}

    Данная программа считывает информацию из файла <<alphabet.txt>>, который содержит строку символов <<ABCDEFGHIJKLMNOPQRSTUVWXYZ>>.
    И при помощи двух буферов посимвольно выводит считанные символы в стандартный поток вывода stdout.  

    Так как отчет печатался на Ч/Б принтере, различия в цветах не видны.
    
    Зеленым цветом показан вывод с помощью первого буфера, синим с помощью второго.
 
    \begin{figure}[ht!]
        \centering{
            \includegraphics[width=1\textwidth]{img/Diagram_1.png}
            \caption{Связь между дескрипторами в первой программе}}
    \end{figure}
    

    В начале функции main open() создает новый файловый дескриптор для открытого
    только на чтение (O\_RDONLY) файла <<alphabet.txt>>, запись в системной таблице открытых файлов. 
    Эта запись регистрирует смещение в файле и флаги состояния файла.
    
    Далее fdopen() создает два указателя на структуру FILE, приведенную выше.
    В данных структурах поле \_fileno будет содержать дескриптор, который вернула
    функция fopen(). Для fs1 и fs2 эти поля будут равны 3.

    Функция setvbuf() изменяет тип буферизации для fs1 и fs2 на
    полную буферизацию, а также явно задает размер буфера 20 байт.

    Далее при первом вызове fscanf() буфер fs1 заполнится полностью, 
    т.е. первыми 20 символами. 
    Значение f\_pos в структуре struct\_file открытого файла увеличится на 20.
    Далее при первом вызове fscanf() для fs2 в buff2 считаются оставшиеся 6 символов,
    начиная с f\_pos (т.к. fs1 и fs2 ссылаются на один и тот же дескриптор fd).

    Далее в цикле поочередно выводятся символы из buff1 и buff2.
    Т.к. в buff2 записались оставшиеся 6 символов, после 6 итерации
    цикла будут выводится символы только из buff1.

    \begin{figure}[ht!]
        \centering{
            \includegraphics[width=0.8\textwidth]{img/res_1.png}
            \caption{Результат работы первой программы}}
    \end{figure}

Первая программа два потока.
    \begin{lstlisting}[language=C]
#include <stdio.h>
#include <fcntl.h>
#include <pthread.h>

#define BUF_SIZE 20

void *run_buffer(void *args)
{
	int flag = 1;
	FILE *fs = (FILE *)args;
	
	while (flag == 1){
		char c;
		if ((flag = fscanf(fs, "%c\n", &c)) == VALID_READED	{
			fprintf(stdout, "thread 2: " "%c\n", c);}}
	return NULL;
}

int main(void)
{
	setbuf(stdout, NULL);
	
	pthread_t thread;
	int fd = open("alphabet.txt", O_RDONLY); 
	
	FILE *fs1 = fdopen(fd, "r");
	char buff1[BUF_SIZE];
	setvbuf(fs1, buff1, _IOFBF, BUF_SIZE); 
	
	FILE *fs2 = fdopen(fd, "r");
	char buff2[BUF_SIZE];
	setvbuf(fs2, buff2, _IOFBF, BUF_SIZE);
	
	int rc = pthread_create(&thread, NULL, run_buffer, (void *)fs2);
	
	int flag = 1;
	while (flag == 1){
		char c;
		flag = fscanf(fs1, "%c\n", &c);
		if (flag == 1){
			fprintf(stdout, "thread 1: " "%c\n", c);
		}
	}
	
	pthread_join(thread, NULL);
	
	return 0;
}
\end{lstlisting}

	\begin{figure}[ht!]
		\centering{
			\includegraphics[width=0.5\textwidth]{img/res_1_2.png}
			\caption{Результат работы первой программы}}
	\end{figure}

\end{task}

\newpage

\begin{task}
    Вторая программа. Один поток.
    \begin{lstlisting}[language=C]
//testKernelIO.c
#include <fcntl.h>
#include <unistd.h> // read, write.

int main()
{
	char c;
	
	int fd1 = open("alphabet.txt", O_RDONLY);
	int fd2 = open("alphabet.txt", O_RDONLY);
	int rc1, rc2 = 1;
	
	while (rc1 == 1 || rc2 == 1){
		char c;
		
		rc1 = read(fd1, &c, 1);
		if (rc1 == 1){
			write(1, &c, 1);}
		
		rc2 = read(fd2, &c, 1);
		if (rc2 == 1){
			write(1, &c, 1);}
	}
	
	write(1, "\n", 1);
	return 0;
}
    \end{lstlisting}

    
    
    \begin{figure}[ht!]
        \centering{
            \includegraphics[width=0.8\textwidth]{img/Diagram_2.png}
            \caption{Связь между дескрипторами во второй программе}}
    \end{figure}
    
    \newpage
    \begin{center}
        \textbf{Объяснение}        
    \end{center}

    В данной программе создается два дескриптора открытого файла при помощи функции open(). 
    %При этом создается две разные структуры struct\_file, описывающие файл.
    В системной таблице открытых файлов создаются две новых записи. %так как f\_pos разные, то чтение будет проходить независимо.
    Далее в цикле поочередно считываются символы из файла и выводятся на экран.
    Т.к. созданы две структуры struct file, то у каждой структуры будет свой f\_pos и смещения в файловых дескрипторах будут независимы, поэтому на экран будут дважды выводится символы одного и того же файла.

    \begin{figure}[ht!]
        \centering{
            \includegraphics[width=0.8\textwidth]{img/res_2.png}
            \caption{Результат работы второй программы}}
    \end{figure}
    % В системной таблице открытых файлов создаются две новые записи, дескрипторы 

    Вторая программа. Два потока.
    \begin{lstlisting}[language=C]
#include <fcntl.h>
#include <unistd.h> // read, write.
#include <pthread.h>
#include <stdio.h>

void read_file(int fd){
	char c;
	while (read(fd, &c, 1))
	{
		write(1, &c, 1);
	}
}
	
void *thr_fn(void *arg){
	int fd = open("alphabet.txt", O_RDONLY);
	read_file(fd);
}

int main()
{
	pthread_t tid;
	
	int fd = open("alphabet.txt", O_RDONLY);
	
	int err = pthread_create(&tid, NULL, thr_fn, 0);
	if (err) {
		printf("Error while create pthread");
		return -1;
	}
	
	read_file(fd);
	pthread_join(tid, NULL);
	
	write(1, "\n", 1);
	return 0;
}
    \end{lstlisting}

    В программе также, как и при реализации с одним потоком, создается
    два файловых дескриптора для открытого файла, записи в системной таблице открытых файлов.
    У каждой записи будет свое смещение f\_pos.
    Т.к. главный поток ждет окончания дочернего, то гарантируется вывод
    всего алфавита дважды.
    Порядок, в котором будут выводится символы алфавита, неизвестен,
    т.к. вывод производится параллельно.


    \begin{figure}[ht!]
        \centering{
            \includegraphics[width=0.8\textwidth]{img/res_2_pthread.png}
            \caption{Результат работы второй программы при двух потоках}}
    \end{figure}

\end{task}

\newpage
\begin{task}
    Третья программа. Один поток.

    \begin{lstlisting}[language=C]
#include <stdio.h>
#include <fcntl.h>
#include <unistd.h>

#include <sys/stat.h>

void info()
{
	struct stat statbuf;
	
	stat("res3.txt", &statbuf);
	printf("inode: %ld\n", statbuf.st_ino);
	printf("st_size: %ld\n", statbuf.st_size);
	printf("st_blksize: %ld\n\n", statbuf.st_blksize);
}

int main()
{
	FILE *f1 = fopen("res3.txt", "w");
	info();
	FILE *f2 = fopen("res3.txt", "w");
	info();
	
	printf("\nfs1 _fileno: %d", f1->_fileno);
	printf("\nfs2 _fileno: %d\n\n", f2->_fileno);
	
	for (char c = 'a'; c <= 'z'; c++)
	{
		if (c % 2)
		{
			fprintf(f1, "%c", c);
		}
		else
		{
			fprintf(f2, "%c", c);
		}
	}
	
	fclose(f1);
	info();
	fclose(f2);
	info();
	
	return 0;
}
    \end{lstlisting}

    В данной программе файл 'res3.txt' открывается 2 раза для записи.
    Выполняется ввод через стандартную библиотеку С (stdio.h). 
    fprintf() - буферизованный ввод/вывод. 
    Буфер создается без нашего явного вмешательства.
    Сначала информация пишется в буфер, а из буфера информация 
    переписывается в файл в результате 3-ех действий:
    \begin{enumerate}
        \item буфер полон;
        \item принудительная запись fflush();
        \item если вызван fclose().
    \end{enumerate}   

    \begin{figure}[ht!]
        \centering{
            \includegraphics[width=1\textwidth]{img/Diagram_3.png}
            \caption{Связь между дескрипторами в третьей программе}}
    \end{figure}

    В нашей программе символы, имеющие нечетный код в таблице ASCII
    записываются в буфер, который находится в дескрипторе f1, 
    в f2 соответственно записываются четные. 
    Таким образом в буфере, который содержится в f1 будут символы: 'acej...', 
    а в f2 'bdfh...'.
    В нашем случае информация из фубера запишется в файл при вызове fclose().
    Т.к. f\_pos независимы у каждого дескриптора файла, то при закрытии файла
    запись будет производиться начиная с начала файла в обоих случаях.
    Таким образом информация, которая будет записана в файл, после первого вызова
    fclose() будет потеряна в результате второго вызова fclose() рис. \ref{ref:res31}.   
   
    \begin{figure}[ht!]
        \centering{
            \includegraphics[width=0.7\textwidth]{img/res_3_1.png}
            \caption{Результат работы третьей программы }
            \label{ref:res31}}
    \end{figure}

    Если поменять вызовы fclose() местами, то будет потеряна информация,
    которая содержится во втором буфере рис. \ref{ref:res32}.
    \begin{lstlisting}[language=C]
    fclose(f2);
    fclose(f1);
    \end{lstlisting}

    \begin{figure}[ht!]
        \centering{
            \includegraphics[width=0.7\textwidth]{img/res_3_2.png}
            \caption{Результат работы третьей программы с другим порядком вызовов fclose()}
            \label{ref:res32}}
    \end{figure}

    \newpage
    С помощью stat после каждого вызова fopen() и fclose()
    показана некоторая информация о файле рис. \ref{ref:info}. 

    \begin{figure}[ht!]
        \centering{
            \includegraphics[width=0.7\textwidth]{img/info_3.png}
            \caption{Информация, полученная при помощи stat}
            \label{ref:info}}
    \end{figure}

    \newpage

    Третья программа. Два потока.
    \begin{lstlisting}[language=C]
#include <stdio.h>
#include <fcntl.h>
#include <pthread.h>
#include <unistd.h>
#include <sys/stat.h>

void info()
{
	struct stat statbuf;
	
	stat("res3.txt", &statbuf);
	printf("inode: %ld\n", statbuf.st_ino);
	printf("st_size: %ld\n", statbuf.st_size);
	printf("st_blksize: %ld\n\n", statbuf.st_blksize);
}

void run_buffer(char c)
{
	FILE *f = fopen("res3_th.txt", "w");
	info();
	
	while (c <= 'z'){
		fprintf(f, "%c", c);
		c += 2;
	}
	
	fclose(f);
	info();
}

void *for_help(void *arg)
{
	run_buffer('a');
}

int main()
{
	pthread_t thread;
	int rc = pthread_create(&thread, NULL, for_help, NULL);
	
	if (rc){
		printf("Error while create pthead");
		return -1;
	}
	
	//sleep(1);
	run_buffer('b');
	
	pthread_join(thread, NULL);
	return 0;
}
    \end{lstlisting}

    В данной программе создается поток.
    Главный поток записывает в файл символы, начиная с 'a', в то время, как созданный
    нами поток записывает символы, начиная с 'b'. 
    Так же как и в приведенной выше программе с одним потоком происходит потеря данных.
    Данные будут записаны из того буфера (который содержится в дескрипторе), для которого
    будет вызван fclose() последним, потому что он перезапишет данные с начала файла.
    Можно принудительно в главном потоке вызвать sleep(), чтобы в файле были данные 
    записанные из главного потока. Тогда результат работы представлен на рис. \ref{ref:res32}.

\end{task}

%\section*{Заключение}

%В ходе выполнения данной лабораторной работы были проанализированы три программы.Открытие файла при помощи open() создает новый файловый дескриптор для открытого файла, запись в системной таблице открытых файлов.У различных дескрипторов открытого файла смещения не зависят друг от друга.Поэтому чтобы избежать потери данных необходимо учитывать, что файл может быть открыт несколько раз.

\end{document}